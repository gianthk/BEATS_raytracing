\section{Introduction}
Its always good to have an introduction, if only to have an example for a section. And here is an example for a reference from the bibtex file (see \cite{einstein}). Its also pretty easy to reference figures (see Figure \ref{fig:examplecernlogo}). \\
\begin{figure}[ht]
\centering
\includegraphics[width=0.5\textwidth]{images/cernlogo.eps}
\caption{\label{fig:examplecernlogo} Example of how to include a figure. This works with all sorts of formats, eps, pdf, png.}
\end{figure}

% this will prevent float objects like figures to be moved past this point in the document.
\FloatBarrier


You also have the option of using colored text, for example \color{blue} this part in blue,  \color{red} this part in red  \color{green} and this part in green, before \color{black} going back to black.  

\begin{enumerate}
\item Everyone loves an enumerated list.
\item If you prefer bulleted lists, see below.
\end{enumerate}

Of course there are always use cases for list with enumerations, and lists with bullets only, which is why it is useful to have examples of both.

\begin{itemize}
    \item Everyone loves a bulleted list.
    \item If you prefer an enumerated list, see above.
\end{itemize}