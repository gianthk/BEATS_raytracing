\documentclass{sesamebeatsnote}
\usepackage[colorinlistoftodos]{todonotes}
\usepackage{placeins}
\usepackage{lscape}
\usepackage{makecell}
\renewcommand\theadalign{cc}
\renewcommand\theadfont{\bfseries}
\usepackage{subcaption}
\usepackage{float}

\title{Report on ML slope error}
\author{BEATS - BEAmline for Tomography at SESAME}
\date{\today}

% Here is the information that will be entered in the title page
\DocAuthors{\href{mailto:gianluca.iori@sesame.org.jo}{G.~Iori}}

% \DocCheckedBy{ D.~Person \\ E.~Person}
\DocCheckedBy{ }
\DocApprovedBy{ }
\EDMSDocNo{BEATS\_ML\_slope\_error}
\EDMSDocId{BEATS-07}
\draftversion{4.3}

\begin{document}
\maketitle

\begin{abstract}
A mirror length of 500mm is considered. Beam profiles at the sample position are shown for different simulated substrates slope errors. A W/B4C stripe with 100 MLs with d-spacing of 3 nm is considered with 45 keV beam energy. 
\end{abstract}

% Make the review table at the bottom of the title page
\vfill
\makereviewtable
\begin{center}
\begin{tabular}[bhp]{|p{0.2\textwidth} p{0.7\textwidth}|}
\hline
\multicolumn{2}{|l|}{\textbf{Version history:}} \\
\multicolumn{1}{|l}{Ver 1.0}  & \multicolumn{1}{l|}{BEATS TDR}  \\
\multicolumn{1}{|l}{Ver 2.0}  & \multicolumn{1}{l|}{Changed substrate size based on feedback from CM and companies}  \\
\multicolumn{1}{|l}{Ver 3.0}  & \multicolumn{1}{l|}{Feedback CM, RB, JRH and RM}  \\
\multicolumn{1}{|l}{Ver 4.0}  & \multicolumn{1}{l|}{Increased ML length to 500mm; W/B4C stripe d-spacing: 3nm}  \\
\multicolumn{1}{|l}{}  & \multicolumn{1}{l|}{Added substrate slope error simulations}  \\
\hline
\end{tabular}
\end{center}
\clearpage

% Short documents dont always need a Table of Content / Figures / Tables, so comment out what is not needed
\begingroup
\color{black}
\tableofcontents
% \listoffigures
% \listoftables
\endgroup
\pagebreak

%%%%%%%%%%%%%%%%%%%%%%%%%%%%%%%%%%%%%%%%%%%%%%%%%%%%%%%%%%%%%%%%%%%%%%%%%%%%%%%%%%%%%%%%%%%%%%%%%%%%%%%%%%%%%
% \section{Introduction}
Its always good to have an introduction, if only to have an example for a section. And here is an example for a reference from the bibtex file (see \cite{einstein}). Its also pretty easy to reference figures (see Figure \ref{fig:examplecernlogo}). \\
\begin{figure}[ht]
\centering
\includegraphics[width=0.5\textwidth]{images/cernlogo.eps}
\caption{\label{fig:examplecernlogo} Example of how to include a figure. This works with all sorts of formats, eps, pdf, png.}
\end{figure}

% this will prevent float objects like figures to be moved past this point in the document.
\FloatBarrier


You also have the option of using colored text, for example \color{blue} this part in blue,  \color{red} this part in red  \color{green} and this part in green, before \color{black} going back to black.  

\begin{enumerate}
\item Everyone loves an enumerated list.
\item If you prefer bulleted lists, see below.
\end{enumerate}

Of course there are always use cases for list with enumerations, and lists with bullets only, which is why it is useful to have examples of both.

\begin{itemize}
    \item Everyone loves a bulleted list.
    \item If you prefer an enumerated list, see above.
\end{itemize}
\subsection{Resources}
\begin{itemize}
    \item BEATS raytracing codes can be found on GitHub \href{https://github.com/gianthk/BEATS_raytracing}{here}.
    \item Figures (e.g. beam profiles and snapshots) are stored \href{https://github.com/gianthk/BEATS_raytracing/tree/master/beam_profiles}{here}.
    \item All power density profiles stored as .CSV files are stored in the \powerprofilesurl.
\end{itemize}
\subsection{Beamline layout}
See beamline drawing and beamline functional layout attached.
%\section{X-ray source - Three Pole Wiggler (3PW)}
\subsection{Wiggler plots}

\subsection{Source size and divergence}
Plots of the horizontal and vertical photon source size and divergence are shown in \ref{fig:3PW_source}. \\.

\begin{figure*}  % spans both columns
\begin{subfigure}{0.45\textwidth}
\includegraphics[width=\linewidth]{figures/beam_snapshots/3PW/sigmaX.png}
% \caption{Network 1}
\end{subfigure}
\hfill % maximize the horizontal distance between the graphs
\begin{subfigure}{0.45\textwidth}
\includegraphics[width=\linewidth]{figures/beam_snapshots/3PW/sigmaXp.png}
% \caption{Network  2}
\end{subfigure}

\bigskip  % some extra vertical whitespace
\begin{subfigure}{0.45\textwidth}
\includegraphics[width=\linewidth]{c.pdf}
\caption{Network  3}
\end{subfigure}
\hfill % maximize the horizontal distance between the graphs
\begin{subfigure}{0.45\textwidth}
\includegraphics[width=\linewidth]{d.pdf}
\caption{Network  4}
\end{subfigure}

\caption{Averages and standard deviations} % Overall figure caption
\end{figure*}

For the power density of the input beam the contribution from two SESAME bending magnets (upstream and downstream of the ID) was considered in addition to the BEATS 3PW. The modified magnetic field profile used for the calculation is shown in Figure \ref{fig:modifiedfieldprofile}. \\
\begin{figure}[ht]
\centering
\includegraphics[width=0.8\textwidth]{images/modified_field_profile.png}
\caption{\label{fig:modifiedfieldprofile} Magnetic field profile modified considering the BEATS source plus two BMs (up and downstream) for calculation of the power load on the crotch absorber.}
\end{figure}
\section{Multilayer mirror 1}
\subsection{Design}
The center of the first multilayer mirror of the BEATS DMM is positioned at 15.165 m from the ID photon source. \\
CINEL will verify the thermal stability of the ML1 with FEA simulations considering the white beam colliding with the mirror at the maximum Bragg angle allowed by the Bragg stage motorization (34.9 mrad). The thermal stability of the cooled mask in front of the ML1 profile shall be also verified.\\

\subsection{Input beam}
A beam snapshot at 15.165 m from source is shown in Figure \ref{fig:snapshot_ML1}.
\begin{figure}[ht]
\centering
\includegraphics[width=0.8\textwidth]{images/WB_snapshot_15.165.png}
\caption{\label{fig:snapshot_ML1} White beam snapshot at 15.165 m from source (center position of ML1).}
\end{figure}

The power density profile at 15.165 m from source is shown in Figure \ref{fig:power_profile_ML1}. Raw data can be found in the \powerprofilesurl. \\
\begin{figure}[ht]
\centering
\includegraphics[width=0.8\textwidth]{images/power_profile_ML1.png}
\caption{\label{fig:power_profile_ML1} Power density profile at 15.165 m from source (center position of ML1).}
\end{figure}

% \bibliography{references}
% \bibliographystyle{plain}

\end{document}
